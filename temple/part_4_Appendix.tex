% 参考文献,此处以 MLA 引用格式为例
\clearpage   %另起一页继续写。这时,你最好使用“\clearpage” 
\begin{thebibliography}{99}
    \bibitem{1} GLOBAL FOREST WHATCH OF AUSTRALIA \\https://www.globalforestwatch.org/topics/fires/?topic=fires\#footer
	\bibitem{2} HU Teng, LIU Zhanjun, LIU Yang, et al. 3D reconnaissance path planning of multiple UAVs. Journal of Systems Engineering and Electronics, 2019, 41(7): 1551-1559.
	\bibitem{3} BASBOUS B. 2D UAV path planning with radar threatening areas using simulated an-nealing algorithm for event detection. The 2018 International Conference on Artificial Intelligence and Data Processing. Malatya, Turkey: IEEE,2018: 1-7.
	\bibitem{4} WANG W F, WU Y C, ZHANG X. Research of the unit decomposing traversal method based on grid method of the mobile robot. Techniques of Automation and Applications, 2013, 32(11): 34-38.
	\bibitem{5} XU Jian, ZHOU Deyun, HUANG He. Multi UAV path planning based on improved ge-netic algorithm. Aeronautical Computing Technique, 2009, 39(4): 43-46.
\end{thebibliography}
\iffalse
\begin{center}
    The following codes are written by \textbf{MATLAB}, which is to collect the data.
\appendix
\end{center}
\fi
\appendix
%\tcblistof[\section]{mcode}{List of MATLAB code}
\section{First Section}
\begin{matlab}{My 1st Code(Written by MATLAB)}{code:labels}
% a comment
clear all, clc
% a comment
x = [1.00; 1.00; 1.00];
beta0 = [1 1 1];
modelfun = 'y ~ k*x^2+b'
mdl = fitnlm(tb,modelfun,beta0)
% a comment
plotResiduals(md1,'fitt111111111111111ed')
%this is so so so so so so so so so so so so so so so so so so so so so so so so so so so long code




















%1
111
\end{matlab}
\begin{matlab}{My 2st Code}{code:llabel}
disp('hello')
\end{matlab}
\begin{matlab}{My 2st Code}{code:labell}
    disp('hello')
\end{matlab}

\section{Python Code Examples}

\subsection{Basic Python Example}
\begin{python}{Simple Python Function}{code:python1}
# Python代码示例
def fibonacci(n):
    """计算斐波那契数列"""
    if n <= 1:
        return n
    else:
        return fibonacci(n-1) + fibonacci(n-2)

# 测试函数
for i in range(10):
    print(f"Fibonacci({i}) = {fibonacci(i)}")
\end{python}

\subsection{Data Analysis with Python}
\begin{python}{Data Processing Example}{code:python2}
import numpy as np
import pandas as pd
import matplotlib.pyplot as plt

# 创建示例数据
data = {
    'x': np.linspace(0, 10, 100),
    'y': np.sin(np.linspace(0, 10, 100))
}
df = pd.DataFrame(data)

# 数据分析和可视化
print("数据统计信息:")
print(df.describe())

plt.figure(figsize=(10, 6))
plt.plot(df['x'], df['y'], 'b-', linewidth=2)
plt.title('正弦函数图像')
plt.xlabel('x')
plt.ylabel('sin(x)')
plt.grid(True)
plt.show()
\end{python}

\section{Use of AI}

This section documents the interactions between the author and AI assistant during the development of this MCM/ICM paper. The conversations are presented in a question-and-answer format to demonstrate how AI was utilized for various aspects of the research.

\subsection{Model Development Assistance}

\begin{tcolorbox}[title=\textbf{Question: How to optimize the UAV path planning algorithm?}, colback=blue!2!white, colframe=blue!30!black, fonttitle=\bfseries]
I'm working on a UAV path planning algorithm for wildfire monitoring. The current approach uses a grid-based method, but I'm encountering issues with computational efficiency when scaling to larger areas. What optimization techniques would you recommend?
\end{tcolorbox}

\begin{tcolorbox}[title=\textbf{AI Response}, colback=green!2!white, colframe=green!30!black, fonttitle=\bfseries]
For UAV path planning optimization in wildfire scenarios, consider these approaches:

1. \textbf{Hybrid algorithms}: Combine genetic algorithms with simulated annealing for better convergence
2. \textbf{Multi-resolution grids}: Use coarse grids for initial planning, then refine locally
3. \textbf{Dynamic programming}: Implement Dijkstra's algorithm with priority queues
4. \textbf{Parallel processing}: Divide the search space for concurrent computation

The key is to balance exploration and exploitation while considering real-time constraints.
\end{tcolorbox}

\subsection{Data Processing and Analysis}

\begin{tcolorbox}[title=\textbf{Question: Data preprocessing for wildfire detection}, colback=blue!2!white, colframe=blue!30!black, fonttitle=\bfseries]
I have satellite imagery data for wildfire detection, but there's significant noise and missing values. What preprocessing steps should I implement before applying machine learning models?
\end{tcolorbox}

\begin{tcolorbox}[title=\textbf{AI Response}, colback=green!2!white, colframe=green!30!black, fonttitle=\bfseries]
Recommended preprocessing pipeline:

\begin{itemize}
\item \textbf{Missing data}: Use interpolation or k-nearest neighbors imputation
\item \textbf{Noise reduction}: Apply Gaussian or median filters
\item \textbf{Normalization}: Scale pixel values to [0,1] range
\item \textbf{Feature extraction}: Consider texture features, thermal gradients
\item \textbf{Data augmentation}: Rotate, flip, and scale images for robustness
\end{itemize}

Also, implement cross-validation to ensure model generalization.
\end{tcolorbox}

\subsection{Mathematical Formulation Assistance}

\begin{tcolorbox}[title=\textbf{Question: Mathematical modeling for fire spread prediction}, colback=blue!2!white, colframe=blue!30!black, fonttitle=\bfseries]
I need to develop a mathematical model for wildfire spread prediction. What are the key factors and equations I should consider?
\end{tcolorbox}

\begin{tcolorbox}[title=\textbf{AI Response}, colback=green!2!white, colframe=green!30!black, fonttitle=\bfseries]
Key factors for wildfire spread modeling:

\begin{align*}
\frac{\partial T}{\partial t} &= \alpha \nabla^2 T + Q \quad \text{(Heat equation with source)}\\
\frac{\partial m}{\partial t} &= -k m \quad \text{(Fuel consumption)}
\end{align*}

Where:
\begin{itemize}
\item $T$: Temperature distribution
\item $m$: Fuel mass density
\item $\alpha$: Thermal diffusivity
\item $Q$: Heat generation rate
\item $k$: Combustion rate constant
\end{itemize}

Consider wind effects, topography, and fuel moisture content as additional parameters.
\end{tcolorbox}

\subsection{Code Implementation Support}

\begin{tcolorbox}[title=\textbf{Question: MATLAB implementation for data visualization}, colback=blue!2!white, colframe=blue!30!black, fonttitle=\bfseries]
I need to create effective visualizations for wildfire data analysis in MATLAB. What plotting techniques and functions would you recommend?
\end{tcolorbox}

\begin{tcolorbox}[title=\textbf{AI Response}, colback=green!2!white, colframe=green!30!black, fonttitle=\bfseries]
For wildfire data visualization in MATLAB:

\begin{lstlisting}[language=Matlab]
% Heat map for temperature distribution
heatmap(temperatureData, 'Colormap', hot);

% Time series for fire progression
plot(time, fireArea, 'r-', 'LineWidth', 2);

% 3D surface for terrain and fire spread
surf(X, Y, elevation, fireIntensity);

% Geographic plotting with mapping toolbox
geoplot(lat, lon, 'r-', 'LineWidth', 2);
\end{lstlisting}

Use subplot for multi-panel figures and customize colormaps for better interpretation.
\end{tcolorbox}

\subsection{Conclusion on AI Usage}

The AI assistant provided valuable support in algorithm optimization, data preprocessing guidance, mathematical formulation, and code implementation. All final decisions, analysis, and interpretations were made by the human author, with AI serving as a collaborative tool for brainstorming and technical guidance.
