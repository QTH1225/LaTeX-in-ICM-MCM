%美赛模板:正文部分

%模板中需要使用Xelatex编译两次,所以可以反复尝试编译
\documentclass[12pt]{article}  % 官方要求字号不小于 12 号,此处选择 12 号字体
% \linespread{1.1} %可选:设置行间距1.1倍
% 本模板不需要填写年份,以当前电脑时间自动生成
% 请在以下的方括号中填写队伍控制号
\usepackage[1234567]{easymcm}  % 载入 EasyMCM 模板文件
\problem{Y}  % 请在此处填写题号
% \usepackage{mathptmx}  % 这是 Times 字体,中规中矩 
\usepackage{palatino}  % mathpazo 这palatino是 COMAP 官方杂志采用的更好看的 Palatino 字体,可替代以上的 mathptmx 宏包
\usepackage{pdfpages}
\usepackage{longtable} %长表格包
\usepackage{tabu}
\usepackage{threeparttable}
\usepackage{listings} %代码排版包
\usepackage{paralist} %压缩列表环境包(紧凑列表)
\usepackage{makecell} %表格换行包
\usepackage{multirow} %表格排版包
%\usepackage{fourier} %使用花里胡哨符号的宏包
%------------------------------------------代码排版包(待研究)
%    \usepackage{minted}  %代码排版包
%    \usepackage{caption} %浮动体包
%    \usepackage{xcolor}
\usepackage{matlab-prettifier} %MATLAB代码排版包
\usepackage{tcolorbox} %美丽盒子包
%    \tcbuselibrary{skins}
%    \tcbuselibrary{minted}
\tcbuselibrary{listings,skins,breakable,xparse}
%    \usemintedstyle{lovelace} %特定代码配色环境
%\tcbuselibrary{breakable}


\usepackage{pgfornament} %花样装饰包(待研究)
\usepackage[ruled,linesnumbered]{algorithm2e} %伪代码排版包
%-------------------------------------------交叉引用实现包
\usepackage{hyperref}
\usepackage{cleveref}

\graphicspath{{img/}}          % 此处{img/}为相对路径,注意加上“/”
 \let\itemize\compactitem
 \let\enditemize\endcompactitem
%-------------------------------------------附录代码排版盒子(起)
 \tcbset{%
 mylist/.style = {%
  colframe = gray,
  colback = white,
  coltitle = red!50!yellow!3!white,
  colbacktitle = white,
  listing only,
  attach boxed title to top center = {yshift = -\tcboxedtitleheight/2},
  enhanced,
  drop fuzzy shadow, % shadow around listings
  left = 6.5mm, % distance between left rule and line number 
  breakable, % enable listing box to break by page
  %enhanced jigsaw, % box not being closed when broken by page
  fonttitle=\small\bfseries\color{black}, % customise font of listing title
  before skip=20pt plus 2pt, % vertical space between listings and text
  after skip=20pt plus 2pt,
 },
 example/.style 2 args = {%
  mylist,
  title = {Listing \thetcbcounter: #1},
  label = {#2},
 },
 }
 \newtcblisting[auto counter, number within = section, list inside = mcode]{matlab}[3][]{%
  listing options = {%
   style = Matlab-editor,
   numbers = left,
   numberstyle = \footnotesize\color{darkgray}\fontfamily{pcr}\selectfont,
   basicstyle = \footnotesize\fontfamily{pcr}\selectfont,
   breaklines = true,  % 启用自动换行
   breakatwhitespace = true,  % 在空格处换行
   postbreak = \mbox{\textcolor{gray}{$\hookrightarrow$}\space}  % 换行指示符
  },
  overlay = {\begin{tcbclipinterior}\fill[blue!15!white] (frame.south west) rectangle ([xshift=5.3mm]frame.north west);\end{tcbclipinterior}},
  example = {#2}{#3}, #1,
 }
 
 % Python代码盒子定义 - 与MATLAB配色统一
 \newtcblisting[auto counter, number within = section, list inside = pcode]{python}[3][]{%
  listing options = {%
   language = Python,
   numbers = left,
   numberstyle = \footnotesize\color{darkgray}\fontfamily{pcr}\selectfont,
   basicstyle = \footnotesize\fontfamily{pcr}\selectfont,
   keywordstyle = \color{darkblue}\bfseries,
   commentstyle = \color{darkgreen},
   stringstyle = \color{red!60},
   showstringspaces = false,
   emphstyle = \color{orange!60},
   emph = {self, cls, True, False, None},
   breaklines = true,  % 启用自动换行
   breakatwhitespace = true,  % 在空格处换行
   postbreak = \mbox{\textcolor{gray}{$\hookrightarrow$}\space}  % 换行指示符
  },
  overlay = {\begin{tcbclipinterior}\fill[blue!15!white] (frame.south west) rectangle ([xshift=5.3mm]frame.north west);\end{tcbclipinterior}},
  example = {#2}{#3}, #1,
 }
%-------------------------------------------附录代码排版盒子(止)

\newcommand{\upcite}[1]{\textsuperscript{\textsuperscript{\cite{#1}}}} 
\crefname{equation}{eq}{eqt} \renewcommand{\cref}[1]{\labelcref{#1}\namecref{#1}} 
\crefname{figure}{fig}{figt} \renewcommand{\cref}[1]{\labelcref{#1}\namecref{#1}}
\crefname{subfigure}{fig}{figt} \renewcommand{\cref}[1]{\labelcref{#1}\namecref{#1}}  
 %载入新命令
%\newcommand{\upcite}[1]{\textsuperscript{\textsuperscript{\cite{#1}}}} %
\title{Input Your Article Title Here \\ if it is too Long}  % 标题

% 如需要修改题头(默认为 MCM/ICM),请使用以下命令(此处修改为 MCM)
%\renewcommand{\contest}{MCM}


%--------------------------------------------文档开始
 \begin{document}

% 此处填写摘要内容
\begin{abstract}

    This is abstract.This section should describe what problem the paper solves, what methods are applied, what results are obtained and summarize them.

    This is the second line abstract.And if you look carefully you can see that the spacing within and between paragraphs is different, which facilitates our reading in paragraphs.

    This is \textbf{the} \textsf{special} \texttt{special} \textsl{special} \textit{special} fonts in abstract.

    \vspace{5pt}  %mm	毫米	1 mm = 2.845 pt   pt 点	1 pt = 0.351 mm
    \textbf{Keywords: Fighting Wildfires; Multi-Objective Optimization; Poisson Distribution; Tabu Search Algorithm; Sensitivity Analysis}

\end{abstract}
\maketitle  % 生成 Summary Sheet

\tableofcontents  % 生成目录
% 正文开始
% Chapter 1: Introduction
\section{Introduction}

\subsection{Problem Background}
Sailing, as an ancient sport, is not only one of the water sports in the Olympic Games, but also reflects the consumption level and consumption preference of different regions through the price of sailing. Studying the prices of sailing boats in different regions can pro-vide reference for sailing brokers and offer buying suggestions for sailing enthusiasts, thus promoting the development and progress of local economy.

Sailboats can be divided into two categories: monomer and catamaran, whose types have different impacts on the price. Therefore, separate study of these two types of sailboats is more conducive to the selection and comparison of sailing brokers and sailing enthusiasts.
\begin{figure}[htbp]
    \centering    
    \subfigure[jeanneau60-Monohulled Sailboats(left)]{				% 图片1([]内为子图标题)						
    \includegraphics[width=0.45\textwidth]{test_1.jpg}}			  % 子图1的图片宽度 不能空行
    \subfigure[excess15-Catamarans(right)]{				% 图片2
    \includegraphics[width=0.45\textwidth]{test_2.jpg}}
	\caption{Comparison of monohull and catamaran} % 图片标题 
\end{figure}
\vspace{-1cm}
\subsection{Restatement of the Problem}
Considering the background information and restricted conditions identified in the problem statement, we need to solve the following problems:

\begin{itemize}
\setlength{\parsep}{0ex} %段落间距
\setlength{\topsep}{2ex} %列表到上下文的垂直距离
\setlength{\itemsep}{1ex} %条目间距
\item Problem 1: We need to build a model to predict the price of different types of sailboats based on some of their characteristics (e.g. beam, drainage, draft, etc.). The accuracy of estimating the price of sailing variety is also discussed.
\item Problem 2: We need to use the above model to discuss the regional consistency of sailing variants and give the actual statistical significance and practical signifi-cance according to the listed price data of sailing vessels in various regions.
\item Problem 3: We need to discuss the role of geographical modelling in the Hong Kong market. By comparing the listing price data, discuss the influence of Hong Kong on the price of monohull and catamaran and whether they are the same.
\item Problem 4\&5: We need to find interesting or meaningful conclusions based on the available data and models, and present our analysis and conclusions to Hong Kong sailing brokers.
\end{itemize}

\subsection{Our Work}
These problems required us to explore the impact of used sailboat parameters and regional effects on sailboat prices. Our work mainly includes:

\begin{itemize}
\setlength{\parsep}{0ex} %段落间距
\setlength{\topsep}{2ex} %列表到上下文的垂直距离
\setlength{\itemsep}{1ex} %条目间距  这三句如果删除就是各条贴在一起
\item First, we use ismissing function and Z-score method to check missing values and outliers.
\item We reduce the dimension of 12 sailboat indexes to 6 indexes, and then use random forest(RF), BP neural network and CNN neural network, and SCA optimization algorithm to optimize the fusion model of the three indexes, and discuss the error of price prediction.
\item For the study of the regional effects of sailing prices, a total of four indicators re-lated to sailing were selected and multiple regression analysis was used. We then calculated the regional effect indicators for the sailing variant to assess the region-al effect of sailing prices.
\end{itemize}

\begin{figure}[htbp]  %h此处,t页顶,b页底,p独立一页,浮动体出现的位置
\centering  %图表居中
\includegraphics[width=.85\textwidth]{test_3.png} %图片的名称或者路径之中有空格会出问题 
\caption{Flow Chart of this Paper's Research} % 图片标题 
\end{figure}
\vspace{-1cm}
%————————————————————————————————————————————————————————————————————————%

\section{Assumptions and Explanations}
Considering that practical problems always contain many complex factors, first of all, we need to make reasonable assumptions to simplify the model, and each hypothesis is closely followed by its corresponding explanation:
\begin{enumerate}[\bfseries \textit{Assumption} 1:]
	\item \textbf{We assume that there is no major financial crisis or credit crisis in the years under study.}\\
	\textbf{\textit{Explanation:}}The financial crisis will cause the global commodity prices to fluctuate greatly, which is not in line with the general law of commodity development and is not conducive to the establishment of the model.
	\item \textbf{We assume that commodities such as sailboats are traded globally on a normal ba-sis without economic sanctions between countries.}\\
	\textbf{\textit{Explanation:}}When economic sanctions occur between countries, competing countries will ma-liciously inflate or downgrade their own goods to hit the rest. This is not in line with the development of commodity prices, and politically induced price changes should also be excluded.
	\item \textbf{We assume the same level of wear and tear on the sailboat.}\\
	\textbf{\textit{Explanation:}}Since the price of a sailboat is influenced by the degree of wear and tear, etc. on the sailboat itself. This is partly a loss due to changes in the sailboat itself and is not considered during the study.
	\end{enumerate}
Additional assumptions are made to simplify analysis for individual sections. These assumptions will be discussed at the appropriate locations.

%————————————————————————————————————————————————————————————————————————%
\section{Notations}
Some important mathematical notations used in this paper are listed in Table 1. 
\begin{table}[htbp]
\begin{center}
\caption{Notations used in this paper}
\begin{tabular}{c c c}
\toprule[2pt]
\multicolumn{1}{m{1.5cm}}{\centering Symbol}
&\multicolumn{1}{m{12.5cm}}{\centering Description }
&\multicolumn{1}{m{1cm}}{\centering Units}\\
\midrule
$\lambda_i$& Principal Components after Dimensionality Reduction&/ \\
$y_i$& Forecasts of Sailboat Prices&/\\
$a_i$& Weights of each Method in Model Fusion&/\\
$\Delta R$&Changes in Sailing Rankings&/ \\
$R^*$ & Indicators of Regional Effects on the Selling Price of Sailboats &/\\
\bottomrule[2pt]
\end{tabular}\label{tb:notation}

 \begin{tablenotes}
        \footnotesize
        \item[*] *There are some variables that are not listed here and will be discussed in detail in each section. %此处加入注释*信息
      \end{tablenotes}
\end{center}
\end{table}
\vspace{-1.2cm}%在\end{table}下加一行\vspace{-1cm} 其中-1的作用是缩短与下方文字距离的 切记!必须是负数

%————————————————————————————————————————————————————————————————————————%
\section{Model Preparation}
\subsection{Data Overview}
For a large amount of data, it is necessary to process and clean the data before building the model. So we first use the ismissing function to find the missing value and get the miss-ing value as follows:
\begin{table}[H]%浮动体的htbp可以强制在原位显示,改参数为H
  \begin{center}
  \fontsize{12pt}{13.8}\selectfont
  \caption{Missing Data in Given Data}
  \resizebox{\textwidth}{!}
  {\begin{tabular}{c c c c c c c c}
  \toprule[2pt]
  \multicolumn{1}{m{3cm}}{\centering \textbf{Type}}
  &\multicolumn{1}{m{2cm}}{\centering \textbf{Make}}
  &\multicolumn{1}{m{2cm}}{\centering \textbf{Variant}}
  &\multicolumn{1}{m{2cm}}{\centering \textbf{Length}}
  &\multicolumn{1}{m{2cm}}{\centering \textbf{Geographic}}
  &\multicolumn{1}{m{2cm}}{\centering \textbf{State}}
  &\multicolumn{1}{m{2cm}}{\centering \textbf{Price}}
  &\multicolumn{1}{m{2cm}}{\centering \textbf{Year}}
  \\ %m后面是列宽
  \midrule
  \multirow{3}*{\makecell[c]{Mono-hulled\\Sailboats}}&Beneteau&Oceanis 54&54&USA&-&\$479,805&2013 \\
  ~ &Delphia&46 cc&46&Europe&-&\$314,606&2013  \\
  ~ &Bavaria&Cruiser 46&46&Europe&-&\$201,640&2014\\ 
  Google Scholar&\multicolumn{7}{c}{This type has NO missing value} \\
  \bottomrule[2pt]
  \end{tabular}}
  \end{center}
\end{table}
\vspace{-1cm}
We remove the above missing values and then use the Z-score method to handle the outliers. Finally, we found that there were no outliers in the original data set.In addition, other data collected in this paper have been processed by the above method.
\subsubsection{Data Collection}
The official website of FEC in Victoria, Australia was queried and lots of data about wildfires were obtained. And other data sources are shown in Table 2.
\upcite{1}\upcite{2}\upcite{3}\upcite{4}\upcite{5}
\begin{table}[H]
  \begin{center}
  \caption{Data and Database Websites}
  \resizebox{\textwidth}{!}
  {\begin{tabular}{c c}
  \toprule[2pt]
  \multicolumn{1}{m{6.5cm}}{\centering \textbf{Database Names}}
  &\multicolumn{1}{m{10cm}}{\centering \textbf{Database Websites} }\\ %m后面是列宽
  \midrule
  GDP of Each Country& https://ourworldindata.org/ \\
  GDP of Some European Countries & https://data.worldbank.org/ \\
  \multirow{2}*{Partial Sailing Parameters}& https://www.sailboat-cruising.com/\\ 
  ~& https://sailboatdata.com/ \\
  \bottomrule[2pt]
  \end{tabular}}
  \end{center}
  \end{table}
  
  \vspace{-1cm}
\subsubsection{Data Screening}
According to the data given, we find that the price of sailboats is not only related to the brand and the time of delivery, but also to the selling area. This is because the actual price of sailing ships is often affected by local economic conditions, so we collected European and American GDP data and visualized it.
\upcite{1}\upcite{2}\upcite{3}\upcite{4}\upcite{5}
\begin{figure}[htbp]
    \centering    
    \subfigure[European GDP Per Capita(left)]{				% 图片1([]内为子图标题)						
    \includegraphics[width=0.45\textwidth]{test_4.png}}			  % 子图1的图片宽度 不能空行
    \subfigure[Part of U.S.GDP per Capita(right)]{				% 图片2
    \includegraphics[width=0.505\textwidth]{test_5.png}} 
	\caption{Data Screening} % 图片标题 
\end{figure}


It is reasonable to predict the price of sailing boats according to the above indexes, which will affect the performance, sailing ability, comfort and service life of sailing boats. Therefore, they are all factors that sailing enthusiasts and sailing brokers need to consider when evaluating the price of sailing boats. At the same time, the importance and weight of these indicators may vary with the type, age, brand and other factors. Therefore, factors such as selling location, brand of sailing boat and year of manufacture were also considered when building the model below.
\begin{table}[H] %浮动体的htbp可以强制在原位显示,改参数为H
  \begin{center}
  \caption{Partial Monohull and Catamaran data}
  \resizebox{\textwidth}{!}
  {\begin{tabular}{c c c c c c}
  \toprule[2pt]
  \multicolumn{1}{m{4cm}}{\centering \textbf{/}}
  &\multicolumn{1}{m{2cm}}{\centering \textbf{Beam length$/m$}}
  &\multicolumn{1}{m{2cm}}{\centering \textbf{Draft/m$^3$}}
  &\multicolumn{1}{m{2cm}}{\centering \textbf{Displacement\\/m$^3$}}
  &\multicolumn{1}{m{2cm}}{\centering \textbf{Sail area/m$^2$}}
  &\multicolumn{1}{m{4cm}}{\centering \textbf{Hull materials}}
  \\ %m后面是列宽
  \midrule
  Bail-4.6&7.2&1.15&11800&116&Carbon fiber \\
  Alliage-AZZURO-53&6.7&1.95&18000&163&Aluminium alloy\\
  \bottomrule[2pt]
  \toprule[2pt]
  \multicolumn{1}{m{4cm}}{\centering \textbf{/}}
  &\multicolumn{1}{m{2cm}}{\centering \textbf{Engine hours/h}}
  &\multicolumn{1}{m{2cm}}{\centering \textbf{Sleeping capacity/person}}
  &\multicolumn{1}{m{2cm}}{\centering \textbf{Headroom\\/m}}
  &\multicolumn{1}{m{2cm}}{\centering \textbf{GDP/\\Trillion\$}}
  &\multicolumn{1}{m{4cm}}{\centering \textbf{-}}
  \\ %m后面是列宽
  \midrule
  Bail-4.6         &2000 &12  &21.87&4.08&- \\
  Alliage-AZZURO-53&4000 &6~8 &17.00&2.56&-\\
  \bottomrule[2pt]
  \end{tabular}}
  \end{center}
\end{table}
\vspace{-1cm}
It is reasonable to predict the price of sailing boats according to the above indexes, which will affect the performance, sailing ability, comfort and service life of sailing boats. Therefore, they are all factors that sailing enthusiasts and sailing brokers need to consider when evaluating the price of sailing boats. At the same time, the importance and weight of these indicators may vary with the type, age, brand and other factors. Therefore, factors such as selling location, brand of sailing boat and year of manufacture were also considered when building the model below.



\section{Layout of Formulas, Graphs and Tables}
\subsection{Layout of Formulas}
We can use the following codes to describe a part of calculation. 

\begin{enumerate}[\bfseries 1.]
    \setlength{\parsep}{0ex} %段落间距
    \setlength{\topsep}{0ex} %列表到上下文的垂直距离
    \setlength{\itemsep}{0ex} %条目间距
    \item A complex line of formulas\dots
    \begin{align}\label{eq:cov}
        C=\left[ \begin{matrix}
	    Cov\left( x_1,x_1 \right)&		Cov\left( x_1,x_2 \right)\\
	    Cov\left( x_2,x_1 \right)&		Cov\left( x_2,x_2 \right)\\
        \end{matrix} \right] ,\ \text{where\ }Cov\left( x_2,x_1 \right) =\frac{\sum_{i=1}^M{\left( x_{1}^{i}-\overline{x_1} \right) ^2}}{M-1}
    \end{align}
    A description of the formula\dots
   
    \item A complex line of formulas\dots
    \begin{align}
      \left[ \begin{array}{c}
	    y_{1}^{i}\\
	    y_{2}^{i}\\
	    \vdots\\
	    y_{k}^{i}\\
        \end{array} \right] =\left[ \begin{matrix}
	    u_{1}^{T}&		\cdots&		\left( x_{1}^{i},x_{2}^{i},x_{3}^{i},\cdots ,x_{N}^{i} \right) ^T\\
	    u_{2}^{T}&		\cdots&		\left( x_{1}^{i},x_{2}^{i},x_{3}^{i}\cdots ,x_{N}^{i} \right) ^T\\
	    \vdots   &		\cdots&		\vdots\\
	    u_{k}^{T}&		\cdots&		\left( x_{1}^{i},x_{2}^{i},x_{3}^{i},\cdots ,x_{N}^{i} \right) ^T\\
        \end{matrix} \right] 
    \end{align}    
    A description of the formula\dots
    \item A complex line of formulas\dots
    \begin{align}
        Con=\frac{\lambda _i}{\sum_{k=1}^p{\lambda _k}},\ \left( i=1,2,\cdots ,p \right) \\
        Con^*=\frac{\sum_{k=1}^i{\lambda _k}}{\sum_{k=1}^p{\lambda _k}},\ \left( i=1,2,\cdots ,p \right) 
    \end{align}
\end{enumerate}

\[\frac{\text{d}x}{\text{d}y}=\frac{\text{d}x}{\text{d}z}\frac{\text{d}z}{\text{d}y}\]

A description of the formula\dots.The \Cref{eq:cov} is important.
\subsection{Layout of Figures}
\begin{figure}[H]
    \centering   %使用\subfigure环境排版三张图也是可以的,但是需要调整图宽度 
    \subfigure[Subfigure1]{				% 图片1([]内为子图标题)						
    \includegraphics[width=0.31\textwidth]{catsquare.jpeg}}			  % 子图1的图片宽度 不能空行
    \subfigure[Subfigure2]{				% 图片2
    \includegraphics[width=0.31\textwidth]{catsquare.jpeg}} 
    \subfigure[Subfigure3]{				% 图片2
    \includegraphics[width=0.31\textwidth]{catsquare.jpeg}} 
	\caption{Subfigures} % 图片标题 
\end{figure}
\vspace{-0.5cm}

\begin{figure}[H] %[H]让图片在文章中的位置就是这段代码的位置
	\centering
		\begin{minipage}[t]{0.3\linewidth} %并排排列图片使用\minipage{}环境,而子图排列使用\subfigure{}环境
			\centering
			\includegraphics[width=0.9\linewidth]{catsquare.jpeg}
            \captionsetup{justification=centering} \caption{Left Figure}
		\end{minipage}%
		\begin{minipage}[t]{0.3\linewidth}
			\centering
			\includegraphics[width = 0.9\linewidth]{catsquare.jpeg}
			\captionsetup{justification=centering} \caption{Middle Figure}
		\end{minipage}%
        \begin{minipage}[t]{0.3\linewidth}
			\centering
			\includegraphics[width = 0.9\linewidth]{catsquare.jpeg}
			\captionsetup{justification=centering} \caption{Right Figure}
		\end{minipage}%
    \vspace{-0.5cm}
\end{figure}


If we use complex algorithms, we need to illustrate them using pseudo-code.

\begin{algorithm}[H]
    \caption{CAT Fusion Pseudo-code}
    \SetAlgoLined
    \KwIn{BIG CATS}
    \KwOut{LITTLE CATS}
    \emph{\textbf{Definition}: $J=Inf$  \& Create $a_{right}$ matrix to record the values of $a_1$, $a_2$ and $a_3$}\;
    \For{$a_3 = 0 to 1(step = 0.01)$}{
        Define the objective function:
        $f\left( a_1 \right) =\min \left\{ \sum{\left( y-\left( a_1x_1+\left( 1-a_3-a_2 \right) x_2+a_3x_3 \right) \right)}^2 \right\} $\;
        Using function fmincon, solve the optimization problem\;
        When the solution is complete, return the resulting $a_1$ and $f(a_1)$\;
        \If{$f(a_1) < J$}{
            $J<--f(a_1)$\;
            Update $a_{right}$ matrix\;
        } 
    }
  \end{algorithm}
\vspace{-0.5cm}

\subsection{Layout of Special Tables}
\begin{table}[H]
    \begin{center}
    \caption{Cute Creatures}\label{table:11}
    \resizebox{\textwidth}{!}
    {\begin{tabular}{c c c c}
    \toprule[2pt]
    \multicolumn{4}{c}{Cute Creatures on Earth}\\
    \midrule %如果表首是单行,则可以在表第二行时再使用\multicolumn{n}{cols}{text}合并表格
    \multicolumn{1}{m{4cm}}{\centering CAT} %此处表格过小,如果不设置行宽会导致文字变大
    &\multicolumn{1}{m{4cm}}{\centering CAT}
    &\multicolumn{1}{m{4cm}}{\centering CAT}
    &\multicolumn{1}{m{4cm}}{\centering CAT}\\
    Cat&Cat&Cat& \makecell[c]{Fat\\Cat} \\
    Cat&Cat&Cat& Cat\\
    Cat&Cat&Cat& Cat\\
    Cat&Cat&Cat& Cat\\
    \bottomrule[2pt]
    \end{tabular}}
    \end{center}
\end{table}
\vspace{-0.5cm}


\section{Conclusions and Interesting Information}
\ldots
\subsection{Useful and Meaningful Conclusions}
\begin{itemize}
    \setlength{\parsep}{0ex} %段落间距
    \setlength{\topsep}{2ex} %列表到上下文的垂直距离
    \setlength{\itemsep}{1ex} %条目间距
    \item As we can see \ldots
    \item As we can see\ldots
    \item As we can see\ldots
\end{itemize}

\section{Sensitivity Analysis}
\dots

\begin{figure}[H]
    \centering
    \begin{tikzpicture}[scale=1.2]
        \draw[-Latex] (0,0,0) -- (3,0,0) node[right] {$x$}; 
        \draw[-Latex] (0,0,0) -- (0,3,0) node[above] {$y$}; 
        \draw[-Latex] (0,0,0) -- (0,0,3) node[below left] {$z$}; 
        \filldraw[fill=gray!30!white, thick] (0,0,2) -- (0,2,0) -- (2,0,0) -- (0,0,2); 
        \draw[densely dashed] (0,0,0) -- (0,2,0); 
        \draw[densely dashed] (0,0,0) -- (0,0,2); 
        \draw[densely dashed] (0,0,0) -- (2,0,0); 
        \node[above right] at (0,0,0) {$O$}; 
    \end{tikzpicture}
    \caption{Three-Dimensional Coordinate System with Triangular Plane}
\label{fig:3d_coordinate}
\end{figure}
\vspace{-0.5cm}

\subsection{Visual Analysis}

\begin{figure}[H]
    \centering
    \begin{subfigure}[t]{0.48\linewidth}
        \centering
        \includegraphics[width=\linewidth]{img/cat.jpeg}
        \captionsetup{justification=centering}
        \caption{Sensitivity Analysis of Parameter $\alpha$}
    \end{subfigure}
    \hspace{0.02\linewidth}
    \begin{subfigure}[t]{0.48\linewidth}
        \centering
        \includegraphics[width=\linewidth]{img/cat.jpeg}
        \captionsetup{justification=centering}
        \caption{Sensitivity Analysis of Parameter $\beta$}
    \end{subfigure}
    \\[0.5cm]
    \begin{subfigure}[t]{0.48\linewidth}
        \centering
        \includegraphics[width=\linewidth]{img/cat.jpeg}
        \captionsetup{justification=centering}
        \caption{Sensitivity Analysis of Parameter $\gamma$}
    \end{subfigure}
    \hspace{0.02\linewidth}
    \begin{subfigure}[t]{0.48\linewidth}
        \centering
        \includegraphics[width=\linewidth]{img/cat.jpeg}
        \captionsetup{justification=centering}
        \caption{Sensitivity Analysis of Parameter $\delta$}
    \end{subfigure}
    \caption{Visualization of Sensitivity Analysis Results}
\label{fig:sensitivity_analysis}
\end{figure}
\vspace{-0.5cm}
Through the four subfigures above, we can clearly observe the sensitivity impact of different parameters on the model output. The top-left figure shows the effect of parameter $\alpha$ variation on the results, the top-right figure displays the sensitivity of parameter $\beta$, the bottom-left figure analyzes the role of parameter $\gamma$, and the bottom-right figure demonstrates the influence of parameter $\delta$.

\begin{figure}[H] %[H]让图片在文章中的位置就是这段代码的位置
	\centering
		\begin{minipage}[t]{0.5\textwidth} %并排排列图片使用\minipage{}环境,而子图排列使用\subfigure{}环境
			\centering
			\includegraphics[width=0.9\textwidth]{catsquare.jpeg}
            \captionsetup{justification=centering} \caption{Left Figure}
		\end{minipage}%
		\begin{minipage}[t]{0.5\textwidth}
			\centering
			\includegraphics[width = 0.9\textwidth]{catsquare.jpeg}
			\captionsetup{justification=centering} \caption{Middle Figure}
		\end{minipage}%
    \vspace{-0.5cm}
\end{figure}
\section{Model Evaluation and Further Discussion}
\subsection{Strengths}

\subsection{Weaknesses \texorpdfstring{$\&$} FFurther Discussion}

\section{Conclusion}
The cites\upcite{1}\upcite{2}\upcite{3}\upcite{4}\upcite{5}
%\input{Memo.tex}
\newpage
% 参考文献,此处以 MLA 引用格式为例
\clearpage   %另起一页继续写。这时,你最好使用“\clearpage” 
\begin{thebibliography}{99}
    \bibitem{1} GLOBAL FOREST WHATCH OF AUSTRALIA \\https://www.globalforestwatch.org/topics/fires/?topic=fires\#footer
	\bibitem{2} HU Teng, LIU Zhanjun, LIU Yang, et al. 3D reconnaissance path planning of multiple UAVs. Journal of Systems Engineering and Electronics, 2019, 41(7): 1551-1559.
	\bibitem{3} BASBOUS B. 2D UAV path planning with radar threatening areas using simulated an-nealing algorithm for event detection. The 2018 International Conference on Artificial Intelligence and Data Processing. Malatya, Turkey: IEEE,2018: 1-7.
	\bibitem{4} WANG W F, WU Y C, ZHANG X. Research of the unit decomposing traversal method based on grid method of the mobile robot. Techniques of Automation and Applications, 2013, 32(11): 34-38.
	\bibitem{5} XU Jian, ZHOU Deyun, HUANG He. Multi UAV path planning based on improved ge-netic algorithm. Aeronautical Computing Technique, 2009, 39(4): 43-46.
\end{thebibliography}
\iffalse
\begin{center}
    The following codes are written by \textbf{MATLAB}, which is to collect the data.
\appendix
\end{center}
\fi
\appendix
%\tcblistof[\section]{mcode}{List of MATLAB code}
\section{First Section}
\begin{matlab}{My 1st Code(Written by MATLAB)}{code:labels}
% a comment
clear all, clc
% a comment
x = [1.00; 1.00; 1.00];
beta0 = [1 1 1];
modelfun = 'y ~ k*x^2+b'
mdl = fitnlm(tb,modelfun,beta0)
% a comment
plotResiduals(md1,'fitt111111111111111ed')
%this is so so so so so so so so so so so so so so so so so so so so so so so so so so so long code




















%1
111
\end{matlab}
\begin{matlab}{My 2st Code}{code:llabel}
disp('hello')
\end{matlab}
\begin{matlab}{My 2st Code}{code:labell}
    disp('hello')
\end{matlab}

\section{Python Code Examples}

\subsection{Basic Python Example}
\begin{python}{Simple Python Function}{code:python1}
# Python代码示例
def fibonacci(n):
    """计算斐波那契数列"""
    if n <= 1:
        return n
    else:
        return fibonacci(n-1) + fibonacci(n-2)

# 测试函数
for i in range(10):
    print(f"Fibonacci({i}) = {fibonacci(i)}")
\end{python}

\subsection{Data Analysis with Python}
\begin{python}{Data Processing Example}{code:python2}
import numpy as np
import pandas as pd
import matplotlib.pyplot as plt

# 创建示例数据
data = {
    'x': np.linspace(0, 10, 100),
    'y': np.sin(np.linspace(0, 10, 100))
}
df = pd.DataFrame(data)

# 数据分析和可视化
print("数据统计信息:")
print(df.describe())

plt.figure(figsize=(10, 6))
plt.plot(df['x'], df['y'], 'b-', linewidth=2)
plt.title('正弦函数图像')
plt.xlabel('x')
plt.ylabel('sin(x)')
plt.grid(True)
plt.show()
\end{python}

\section{Use of AI}

This section documents the interactions between the author and AI assistant during the development of this MCM/ICM paper. The conversations are presented in a question-and-answer format to demonstrate how AI was utilized for various aspects of the research.

\subsection{Model Development Assistance}

\begin{tcolorbox}[title=\textbf{Question: How to optimize the UAV path planning algorithm?}, colback=blue!2!white, colframe=blue!30!black, fonttitle=\bfseries]
I'm working on a UAV path planning algorithm for wildfire monitoring. The current approach uses a grid-based method, but I'm encountering issues with computational efficiency when scaling to larger areas. What optimization techniques would you recommend?
\end{tcolorbox}

\begin{tcolorbox}[title=\textbf{AI Response}, colback=green!2!white, colframe=green!30!black, fonttitle=\bfseries]
For UAV path planning optimization in wildfire scenarios, consider these approaches:

1. \textbf{Hybrid algorithms}: Combine genetic algorithms with simulated annealing for better convergence
2. \textbf{Multi-resolution grids}: Use coarse grids for initial planning, then refine locally
3. \textbf{Dynamic programming}: Implement Dijkstra's algorithm with priority queues
4. \textbf{Parallel processing}: Divide the search space for concurrent computation

The key is to balance exploration and exploitation while considering real-time constraints.
\end{tcolorbox}

\subsection{Data Processing and Analysis}

\begin{tcolorbox}[title=\textbf{Question: Data preprocessing for wildfire detection}, colback=blue!2!white, colframe=blue!30!black, fonttitle=\bfseries]
I have satellite imagery data for wildfire detection, but there's significant noise and missing values. What preprocessing steps should I implement before applying machine learning models?
\end{tcolorbox}

\begin{tcolorbox}[title=\textbf{AI Response}, colback=green!2!white, colframe=green!30!black, fonttitle=\bfseries]
Recommended preprocessing pipeline:

\begin{itemize}
\item \textbf{Missing data}: Use interpolation or k-nearest neighbors imputation
\item \textbf{Noise reduction}: Apply Gaussian or median filters
\item \textbf{Normalization}: Scale pixel values to [0,1] range
\item \textbf{Feature extraction}: Consider texture features, thermal gradients
\item \textbf{Data augmentation}: Rotate, flip, and scale images for robustness
\end{itemize}

Also, implement cross-validation to ensure model generalization.
\end{tcolorbox}

\subsection{Mathematical Formulation Assistance}

\begin{tcolorbox}[title=\textbf{Question: Mathematical modeling for fire spread prediction}, colback=blue!2!white, colframe=blue!30!black, fonttitle=\bfseries]
I need to develop a mathematical model for wildfire spread prediction. What are the key factors and equations I should consider?
\end{tcolorbox}

\begin{tcolorbox}[title=\textbf{AI Response}, colback=green!2!white, colframe=green!30!black, fonttitle=\bfseries]
Key factors for wildfire spread modeling:

\begin{align*}
\frac{\partial T}{\partial t} &= \alpha \nabla^2 T + Q \quad \text{(Heat equation with source)}\\
\frac{\partial m}{\partial t} &= -k m \quad \text{(Fuel consumption)}
\end{align*}

Where:
\begin{itemize}
\item $T$: Temperature distribution
\item $m$: Fuel mass density
\item $\alpha$: Thermal diffusivity
\item $Q$: Heat generation rate
\item $k$: Combustion rate constant
\end{itemize}

Consider wind effects, topography, and fuel moisture content as additional parameters.
\end{tcolorbox}

\subsection{Code Implementation Support}

\begin{tcolorbox}[title=\textbf{Question: MATLAB implementation for data visualization}, colback=blue!2!white, colframe=blue!30!black, fonttitle=\bfseries]
I need to create effective visualizations for wildfire data analysis in MATLAB. What plotting techniques and functions would you recommend?
\end{tcolorbox}

\begin{tcolorbox}[title=\textbf{AI Response}, colback=green!2!white, colframe=green!30!black, fonttitle=\bfseries]
For wildfire data visualization in MATLAB:

\begin{lstlisting}[language=Matlab]
% Heat map for temperature distribution
heatmap(temperatureData, 'Colormap', hot);

% Time series for fire progression
plot(time, fireArea, 'r-', 'LineWidth', 2);

% 3D surface for terrain and fire spread
surf(X, Y, elevation, fireIntensity);

% Geographic plotting with mapping toolbox
geoplot(lat, lon, 'r-', 'LineWidth', 2);
\end{lstlisting}

Use subplot for multi-panel figures and customize colormaps for better interpretation.
\end{tcolorbox}

\subsection{Conclusion on AI Usage}

The AI assistant provided valuable support in algorithm optimization, data preprocessing guidance, mathematical formulation, and code implementation. All final decisions, analysis, and interpretations were made by the human author, with AI serving as a collaborative tool for brainstorming and technical guidance.




\end{document}